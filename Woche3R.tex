Weit unter mir rauscht das Wasser. Um mich herum erheben sich hell erleuchtete Bürogebäude in den tiefschwarzen Himmel. Zu meiner rechten eine Art Festung aus dem 13. Jahrhundert und weiter entlang des Flusses ein weit ausgedehnter Gebäudekomplex, mit einem beeindruckenden Turm, an dem eine gigantische Uhr hängt. Es ist acht Uhr abends und ich bin natürlich in London. Der Fluss unter mir ist die Themse, ich selbst stehe auf der Tower Bridge und denke nach.

Am Anfang war meine neue Freiheit vor allem eines: Aufregend! Schon der erste Moment, die erste Entscheidung, die zu treffen war, als ich auf der Straße stand: Links oder Rechts? Ich entschied mich für links. Als nächstes kam ich auf eine breite Straße und an deren Ende: Die Lübecker Kirchtürme. Nun wusste ich, wohin ich gehen wollte. Natürlich war ich auch nicht wirklich verloren — schließlich habe ich sämtliche Karten der Umgebung gespeichert und kann stets per Satellitensignal ermitteln, wo ich gerade bin. Ich marschierte also in Richtung Lübecker Innenstadt, im strahlenden Sonnenschein. Nur die Menschen irritierten mich. Wie sie mich anstarrten und auf mich zeigten. Einige kamen neugierig näher, andere wichen auf die andere Straßenseite aus. Ein kleiner Junge ist sogar schreiend vor mir weggelaufen. Und einmal habe ich ein seltsames Gespräch belauscht. Ein Mann zeigte auf mich und sagte zu seinem Sohn: "`Guck mal, Dominik, der sieht aus wie C3-PO!"'

Der Junge schien nicht sonderlich begeistert und grummelte nur: "`Gar nicht wahr. Bei Star Wars sieht der ganz anders aus. Du hast ja keine Ahnung."' Ich versuchte mir von derlei Reaktionen nicht den Tag verderben zu lassen. Ich ignorierte die Menschen weitgehend und verbrachte einen ganzen Tag damit, Lübeck zu erkunden.

Als nächstes zog es mich ans Meer, also lief ich bis nach Travemünde. Dort schmuggelte ich mich schließlich auf eine Fähre und landete in Malmö. Aber auch die Schweden sahen mich komisch an. Sie zeigten auf mich, sie riefen mir wütende Worte zu, bezeichneten mich als Freak oder drehten um, wenn sie mich sahen. Dann trauten sich einige, auf mich zuzugehen. Ich versuchte es mit einem freundlichen "`Hej!"', doch anstatt der Gesprächsetiquette zu folgen und mich ebenfalls zu begrüßen, stellten sich um mich herum auf und machten Fotos. Einer sagte "`Cooles Kostüm, Kumpel. Aber bis Fasching is' eigentlich noch ne Weile"' und schubste mich in einen Busch. Lachend zog er mit seinen Freunden davon. Da hielt ich es nicht mehr aus. Ich nahm den schnellsten Weg aus der Stadt, lief über Felder und endlich in den Wald, der mir Schutz vor den Blicken der Menschen bot. Ich setzte mich auf einen Baumstumpf und dort blieb ich, für Stunden, einen ganzen Tag lang. Ich hatte die Menschen ganz anders berechnet. Nicht so gleichgültig wie Jona. Und nicht so misstrauisch und gehässig wie die Passanten auf den Straßen. Aber es musste doch noch andere geben? Solche mit Anteilnahme und Empathie, solche die tolerant und offenherzig sind? Ich verließ mein Versteck, zog zurück in die Stadt. So einfach wollte ich nicht aufgeben. Ich würde sie finden, wenn es sie irgendwo gab, dann würde ich sie finden. Aber von Malmö hatte ich wirklich genug, daher schlich ich zum Hafen und schmuggelte mich wieder auf ein Schiff. Nach einer Weile erreichten wir London. Gemeinsam mit mehreren Hundert Containern ging ich von Bord.
Und seitdem bin ich hier. Gestern immerhin hat mich eine junge Frau angesprochen. Die hat gefragt, wer ich bin, was ich bin. Und was ich alles kann. Sie war etwas enttäuscht, dass ich keine Laserstrahlen aus den Augen schießen lassen kann. Dafür schien es sie zu faszinieren, dass es einen Roboter wie mich überhaupt gibt. Einen der frei seine Entscheidungen treffen kann. Ich hab ihr auch von Jona erzählt; sie meinte es sei eine erstaunliche Leistung gewesen, mich zu bauen. Vielleicht bin ich zu hart zu Jona gewesen, schließlich arbeitet er ja wirklich viel. Vielleicht sollte ich doch wieder zurück? Es gibt bestimmt hier ein Schiff nach Hamburg.

Nachdenklich blicke ich hinunter auf die Themse. Aber die kann mir leider auch nicht sagen, was ich tun soll. Plötzlich höre ich hinter mir eine Stimme: "`Wow, das wäre genau das richtige für Luke!"' Und ehe ich mich versehe, hat mich jemand in einen Sack gestopft. Alles ist dunkel und es riecht muffig. Nur den Lärm der Londoner Straßen kann ich noch hören. Ich versuche zu protestieren, zu schreien, doch auf meine blecherne Stimme reagiert natürlich keiner. Dann Motorengeräusch — anscheinend ist mein Entführer mit mir in ein Taxi gestiegen. Immerhin habe ich einigermaßen brauchbaren Satellitenempfang und weiß daher wo ich bin. Mir wird schnell klar, dass wir raus aus der Stadt fahren, irgendwo ins nirgendwo. Dort werde ich zum Glück aus diesem Sack befreit. Wir sind ganz allein auf einem einsamen Feld, mein Entführer und ich, und das Taxi braust schon in die Dunkelheit davon. Erst jetzt habe ich Gelegenheit, mir meinen Entführer genauer anzusehen. Roter Mantel, rote Mütze, weißer Bart, weißes Haar — ich traue meinen Augen kaum, es ist der Weihnachtsmann. Und hinter ihm steht im Matsch ein Schlitten, gezogen von neun Rentieren. "`Nun komm, steig schon auf. Hier kannste sowieso nirgends fliehen. Und wenn du nicht gehorchst, kommst du wieder in den Sack!"'

Ich bin völlig erschüttert, aber ich gehorche und kurze Zeit später hebt der Schlitten ab. In rasantem Tempo überfliegen wir England und dann den atlantischen Ozean.

"`Zum Glück kann das Teil fliegen, sonst hätt' ich echt 'n Problem, so wenig Schnee, wie's heutzutage gibt... Der Nordpol ist mir auch schon unter den Füßen weggeschmolzen! Aber da wo wir jetz' hinfliegen gibt's noch welchen, hoch oben in den Rocky's!"'

Ich verkneife mir eine Antwort, und das obwohl der Weihnachtsmann abgesehen von der Frau in London der einzige ist, der je mit mir reden wollte. Zu erschüttert bin ich, dass das erträumte Idol der Kinder völlig real und furchtbar gemein ist.

Wir fliegen tatsächlich in Richtung Rocky Mountains. Irgendwo zwischen drei Schnee bedeckten Gipfeln ist das Anwesen des Weihnachtsmanns — eine gemütliche Hütte und ein schäbiges Lager. Ich lande natürlich in dem schäbigen Lager, klar. Darin ist es furchtbar kalt — aber immerhin bin ich nicht mehr allein: Hier stapelt sich Spielzeug auf Spielzeug auf Spielzeug.Teddybären, Puppenhäuser. Plastikautos, Plastikküchen, Plastikschlitten. Playmobil, Bauklötze. Und Handys, Computer, Gameboys. Ich bin fasziniert, durchwandere das ganze Lager. Mit dem Spielzeug kann ich leider nicht kommunizieren, dafür versuche ich es (über digitale Signale) mit den technischen Geräten. Ich bin aber schnell enttäuscht: Die Rechenleistung ist teilweise gar nicht schlecht, auch der Speicher nicht  — doch die Programme sind derart rudimentär und simpel, dass vernünftige Kommunikation kaum möglich ist. Trotzdem mache ich mir die Mühe, alles genau abzusuchen. Es muss hier doch irgendwas geben, womit ich mich aus meiner misslichen Lage befreien kann.